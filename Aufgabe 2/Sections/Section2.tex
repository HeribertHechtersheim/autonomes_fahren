\section{Umrechnung der Pixelkoordinaten in Weltkoordinaten}

Nachdem wir die Eckpunkte des gesuchten Rechtecks im Bild der Kamera gefunden haben, möchten wir nun die Orientierung $^A\xi_B$ des vorausfahrenden Autos $B$ gegenüber dem Auto $A$ bestimmen.
Die Koordinatensysteme $\{A\}/\{B\}$ seien dafür mittig zwischen den Hinterrädern auf Höhe des Bodens plaziert, wobei die x-Achse in Fahrtrichtung und die z-Achse in den Himmel zeigt. Die Orientierung des Kamerakoordinatensystems $\{C\}$ zum Auto $A$ ist bekannt und wird mit $^A\xi_C$ bezeichnet. Außerdem ist die intrinsische Matrix $K$ der Kamera bekannt und hat folgende Form:
\begin{equation*}
K= \begin{pmatrix} f/\rho_\omega & 0 & u_0 \\ 0 & f/\rho_h & v_0 \\ 0 & 0 & 1 \\ \end{pmatrix}
\end{equation*}
Sei nun die homogene Pixelkoordinate $\tilde{p} = (u,v,1)^T$ eines Punktes gegeben und dessen homogene Koordinate $^A\tilde{P} = \lambda(X,Y,Z,1)^T, \lambda \in \mathbb{R}$ bezüglich des Autos $A$ gesucht. Durch $K$ und homogener Koordinatentransformationsmatrix ${^CT_A}$ von $^C\xi_A$ lässt sich folgende Gleichung aufstellen:
\begin{equation*}
\label{eq:problem}
	\tilde{p} = K \begin{pmatrix} 1 & 0 & 0 & 0 \\ 0 & 1 & 0 & 0 \\ 0 & 0 & 1 & 0 \\ \end{pmatrix} {^CT_A} ^A\tilde{P}
\end{equation*}
Durch die Rechtecksmatrix ist die Bestimmung von $^A\tilde{P}$ wegen dem unbekannten Faktor~$\lambda$ nicht eindeutig möglich:
\begin{equation}
\label{eq:^A_P}
	{^AT_C} \begin{pmatrix} K^{-1}\tilde{p} \\ \lambda \end{pmatrix} = {^A\tilde{P}}
\end{equation}
Deswegen benutzen wir zusätzlich die Annahme, dass die Höhe $Z$ des Punktes bekannt ist und können die dritte Zeile der Gleichung nach $\lambda$ umformen:
\begin{equation*}
	\lambda = \frac{e_3^T{^AT_C} K^{-1}\tilde{p}}{Z - {(^AT_C)}_{3,4}}
\end{equation*}
Dies setzte vorraus, dass die Höhe $Z$ des Punktes nicht mit der Höhe der Kamera ${(^AT_C)}_{3,4}$ übereinstimmt. Nachdem $\lambda$ errechnet worden ist, kann $^A\tilde{P}$ durch Gleichung~\ref{eq:^A_P} berechnet werden.

Unter der Annahme, dass das Auto A in der gleichen Ebene wie das Auto B liegt und wir zwei Punkte $\tilde{p}_1$ und $\tilde{p}_2$ mit der Kamera erkannt haben, von denen wir die Koordinaten $^B\tilde{P}_1$ und $^B\tilde{P}_2$ bezüglich $\{B\}$ kennen, kann nun $^A\xi_B$ bestimmt werden. In unserer Implementierung benutzen wir hierfür die unteren beiden Eckpunkte des roten Rechtecks.

Um die Annahme der gleichen Ebene zu vermeiden, kann auch ein Gleichungssystem für mehrere Punkte durch Gleichung~\ref{eq:problem} und deren Koordinaten bezüglich $\{B\}$ ausgestellt werden. Dies erfordert aber eine aufwändigere Betrachtung der eindeutigen Lösbarkeit.
% p = p(:);
% g = K\[p;1];
% lambda = T(3,1:3)*g/(z-T(3,4));
% P = T(1:3,:)*[g/lambda; 1];

