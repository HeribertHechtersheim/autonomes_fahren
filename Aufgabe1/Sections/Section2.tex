\section{Massenträgheitsmoment}
\subsection{Theorie}
Das Trägheitsmoment kann experimentell durch einen Pendelversuch bestimmt werden. 
Für eine harmonische Schwingung kann die Auslenkung $x$ mit der Kreisfrequenz $\omega$ als Funktion der Zeit dargestellt werden:
\[ \frac{d^2x}{dt^2}+\omega^2x=0.\]
Das Drehmoment, das auf den Körper wirkt beträgt 
\[M=-mgd = -mglsin(\varphi).\]
In Verbindung mit \(M(t) = I\alpha(t)\) gilt \[ \frac{d^2\varphi}{dt^2}+\frac{mgl}{I}sin(\varphi)=0.\]
Für kleine Winkel und der damit verbundenen Annahme \(sin(\varphi)\approx \varphi\) gilt \[\frac{d^2\varphi}{dt^2}+\frac{mgl}{I} \varphi =0.\]
Damit ist  die Schwingungsdauer $T$  
\[T=\frac{2\pi}{\omega} = 2\pi \sqrt{\frac{I}{mgl}}\] 
und das Trägheitsmoment $I$ 
\[I=\Big(\frac{T}{2\pi}\Big)^2mgl.\]

\subsection{Praxisversuch}
Das für den Pendelversuch verwendete Pendel besteht aus einer Holzplatte und einem Gestell. Das kombinierte Trägheitsmoment für Platte und Gestell bezüglich des gemeinsamen Schwerpunktes beträgt $0.057 kg m^2.$
Die Schwingungsdauer $T$ mit montiertem Auto beträgt $1.3749s$, das Gesamtgewicht aus Holzplatte, Gestell und Auto ergibt sich zu $m_{ges}=0.846+1.152+2.26 = 4.258kg$. Die Pendellänge $l$ ist der Abstand  vom gemeinsamen Schwerpunkt zur Drehachse und beträgt  $0.383m.$   Daraus  ergibt sich ein Trägheitsmoment von 
\[I_{ges} = \Big(\frac{1.3749 s}{2\pi}\Big)^2 \cdot 4.258 kg \cdot 9.81\frac{m}{s^2} \cdot 0.383m = 0.7662 kg m^2.\]
Dieses ist das Trägheitsmoment  des kombinierten Körpers (Holzplatte, Gestell und Auto) bezüglich der Drehachse. Mit Hilfe des Steiner'schen Satzes kann die Bezugsachse in den gemeinsamen Schwerpunkt verschoben werden: 
\[I_{ges,SP} = I_{ges} + m_{ges} \cdot l^2  \hspace{5mm} \rightarrow  \hspace{5mm} 0.7662 kg m^2+ 4.258 kg \cdot (0.383m)^2 = 1.39kgm^2.\] 
Das Trägheitsmoment des Autos bezüglich des Schwerpunkts berechnet sich aus der Differenz des Gesamtträgheitsmoment und des Trägheitsmoment der Aufhängung:
\[1.39kgm^2-0.057kgm^2 = 1.333 kgm^2.\]
